\documentclass[11pt]{article}
\usepackage[utf8]{inputenc}
\usepackage[english]{babel}

%Import the natbib package and sets a bibliography  and citation styles
\usepackage{natbib}
\bibliographystyle{bathx}
\usepackage{parskip} % No paragraph indents%
\setcitestyle{authoryear,open={(},close={)}} %Citation-related commands

\usepackage[a4paper,margin=2.5cm]{geometry}
\renewcommand{\familydefault}{\sfdefault} % use sans serif by default
\usepackage{graphicx}

\usepackage{url}
\usepackage{comment}
% Highlight text, \hl{}  the default colour will be yellow
\usepackage{soul}
\usepackage{xcolor}

\usepackage{hyperref} % Provides hyperlinks to sections automatically
% default formatting for links
\hypersetup{
    colorlinks=true,
    linkcolor=blue,
    filecolor=violet,      
    urlcolor=violet,
    citecolor= teal,
    pdftitle={Overleaf Example},
    pdfpagemode=FullScreen,
}

\setcounter{tocdepth}{1} % control the depth of the table (1 - Chapters and Sections, 2 - adds subsections)

\title{Agile Processes and Reflections}
\author{}

\begin{document}

\maketitle
\tableofcontents % insert the below for a table of contents
\listoffigures % it is useful to include a list of figures at least in CW2 or if you end up using a lot in CW1

\section{Introduction to Agile}
This is a brief description and introduction of AGILE.

\subsection{Motivation and What Existed Prior}
This is section where we talk about the precursors to AGILE and their limitations.
We should also set the stage for AGILE here by positioning it as a methodology that makes up for shortcomings in
its precursors

\subsubsection{Waterfall}
A brief description of Waterfall, along with its shortcomings

\subsubsection{Iterative Waterfall}
A brief description of Iterative Waterfall, along with its shortcomings

\subsubsection{V-Model}
A brief description of the V-Model, along with its shortcomings

\subsubsection{Re-use Pattern}
A brief description of the Re-use Pattern, along with its shortcomings

\subsubsection{Formal Pattern}
A brief description of the Formal Pattern, along with its shortcomings

\subsubsection{Evolutionary Pattern or Prototyping}
A brief description of the Evolutionary Pattern, along with its shortcomings

\subsubsection{Spiral Model}
A brief description of the Spiral Model, along with its shortcomings

\subsection{AGILE Processes and Methodologies}
In this section, we deep dive into AGILE and its specific manifestations, while giving a critical overview.

\subsubsection{TDD}
Brief description of TDD and critical analysis

\subsubsection{Extreme Programming}
Brief description of Extreme Programming and critical analysis

\subsubsection{SCRUM}
Brief description of SCRUM and critical analysis

\section{Reflections on Pairworking}

\subsection{Tools and Practices}
Talk about the formal practices and tools we used to work on this project

\subsubsection{Git and Github}
Talk about how we used git and github and how it affected our workflow

\subsubsection{Teams and Teams Documents}
Talk about how we used teams and teams documents and how it affected our workflow

\subsubsection{Pair Programming}
Talk about the workflow that was established in order to complete the task. Talk about how viable it was and how it impacted your work.

% \paragraph{Smaller still?}
% some arbitrary text, for a picture of agile methodology see Figure~\ref{fig:agile}

% % The following syntax creates a bullet point list
% \begin{itemize}
% \item Reason why agile great
% \item Reason why agile bad
% \item Reason why agile \textbf{neutral}
% \end{itemize}

% An example of a numbered list, again you can do this in 'Rich Text' style
% \begin{enumerate}
% \item First point in numbered list
% \item Second point in numbered list
% \end{enumerate}

% You may want to highlight some text, \hl{the default colour for this is yellow}

% \includecomment{You can use this syntax as an alternative to the percentage sign for writing big blocks of comment.}
% the syntax you need in order to include graphics
% centering prevents the image from being left aligned
% Width determines how big the image is
% You can use a label so that you can reference the image within your text
% the [!h] makes the image sit within the next (experiment with removing this)
% \begin{figure}[!ht]
% \centering
% \includegraphics[width=0.7\textwidth]{agilePic}\caption{Agile methodology picture.}\label{fig:agile} 
% \end{figure}

% \subsection{How to cite}

% Agile is great according to \citet{schwaber:2002}. To achieve this style of citation include the syntax \verb|\citet{}|



% Agile is great \citep{schwaber:2002}. To include this style of citation, use \verb|\citep{}|.

% An example of a citation from an online resource is \cite{WinNT}.

% \section{This is my evaluation of methodology X}

%An example table
% \begin{table}[htb]
% \caption{An example table}
% \bigskip
% \begin{center}
% \label{Example-Table}
% \begin{tabular}{|l|l|}
% \hline
% Items & Values \\
% \hline
% \hline
% Item 1 & Value 1 \\
% Item 2 & Value 2 \\
% \hline
% \end{tabular}
% \end{center}
% \end{table}

% \section{This is my evaluation of tool Y}
% \section{This is my evaluation of practice Z}




\bibliography{thisismybib}
\end{document}


